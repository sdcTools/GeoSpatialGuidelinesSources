\section{Auxiliary Methods} \label{sec:aux}

\subsection{Binning values} \label{sec:aux_bin}

It can be helpful for protection to sort counts or continuous values into \emph{discrete class bins}. For example, in a map that shows local unemployment rate at the grid cell level, one may show for a cell not the value 12.5\%, but instead the class label `10-15\%'. The effect is similar to rounding, in that the user can only infer an interval for the true value; see \citet[5.5]{HundepoolEtAl2024}. Secondary disclosure risks are thereby also similar to deterministic rounding. 

\begin{tcolorbox}[breakable]
\emph{Example}:
Suppose we know from another map that the grid cell with class label `10-15\% unemployed' has a total of 8 persons in it. This reveals that 1 person is unemployed, since 1 in 8 is the only count that leads to a fraction between 10-15\%.
\end{tcolorbox}

For such reasons binning alone will often not suffice to create a safe geo-referenced data product. Instead, it is often advisable to use binning as an auxiliary method for post-processing the output from another SDC method. For example, \citet{JongeWolf2016} suggest discretising the density map from spatial smoothing (see section \ref{sec:methods_smooth}). Some general recommendations to increase the protection from binning are:
\begin{itemize}
    \item Choosing large enough uncertainty intervals.
    \item Curating the release program. For instance, the disclosure from unemployment rates in the example above relied on an additional release of true counts for the same grid.
    \item Avoiding intersecting or custom bins. For instance, releasing the same grid cell in two maps, where it is labeled in one `10-15\% unemployed' and `12\% or more unemployed' in the other, allows inferring that the true value must be in the interval 12-15\%, which is sharper than any of the two.
\end{itemize}

\subsection{Suppressing cell values} \label{sec:aux_suppr}

When aggregates for small geographical regions, such as grid cells, are considered unsafe for publication, the corresponding value may be suppressed -- meaning the value is kept from the publication and replaced by a suppression symbol, like a dot or `$\times$'. Thus replacing unsafe values is called \emph{primary suppression}. 
%Suppressed values are typically marked with a specific symbol in tabular representation, and with a specific color in choropleth maps to distinguish them from empty cells.

If aggregates are published in nested area systems (for instance 100m $\times$ 100m grid cells and 200m $\times$ 200m grid cells), then the suppressed value may still be disclosed by subtraction. For instance, if from the four nested cells that make up a large cell, a single one is suppressed, one would only need to subtract the value of the other three from the value of the large cell. To prevent this, at least \emph{two} suppressed cells are required within the larger cell. The value of the large cell is equivalent to a `column total' (or sub total) in a table. At the same time there may be a `row total' along the subject domain: We may publish, for instance, counts of total population, native population and non-native population per grid cell. If we suppress the value of `non-native' for an area, we must suppress that of `native' as well in order to prevent disclosure by subtraction. The problem of \emph{secondary suppression} from statistical disclosure control for tables therefore applies. The reader is referred to \citet[ch. 4]{HundepoolEtAl2024} for details.

Some further remarks on cell suppression in the geographic context:
\begin{itemize}
    \item Grid products for whole countries or similarly detailed geographic breakdowns may require a large number of cells to be suppressed. 
    \item The information lost is likely to be concentrated in low-density regions and at the edges of clusters (see section \ref{sec:ident_geochar}).
    \item By choosing a larger grid size, one will typically have to suppress fewer cells, but the information lost per suppression will be higher.
    \item One can also use cell suppression on top of another disclosure control method to improve safety. Then the suppression can often be relaxed somewhat (for instance no secondary suppression).
    \item Suppression may be used to resolve geographical differencing issues, as described in \ref{sec:risk_diff}, though finding the right suppression pattern for that purpose can prove challenging.
\end{itemize}

\newpage